\documentclass[a4paper,twoside,10pt,french]{scrartcl}
\usepackage[T1]{fontenc}
\usepackage{titlesec}
\usepackage{ulem}
\usepackage[dvipsnames]{xcolor}
\usepackage{color}
\usepackage{babel}
\usepackage{helvet}
\usepackage[utf8]{inputenc}
\usepackage[T1]{fontenc}
\usepackage{geometry}
\usepackage{graphicx}
\title{33 page 286 - C7}
\date{}
 \renewcommand*\familydefault{\sfdefault}
\geometry{a4paper}
% -------------------------------------------------------------------------------------------------
% ----------- Création des commandes de couleur pour les titres ----------
% -------------------------------------------------------------------------------------------------
\newcommand{\sectionred}[1]{{\color{red}{\uuline{\color{black}#1}}}}
\newcommand{\sectiongreen}[1]{{\color{ForestGreen}{\uuline{\color{black}#1}}}}
\newcommand{\sectionblue}[1]{{\color{NavyBlue}{\uuline{\color{black}#1}}}}
\begin{document}
\maketitle
% -------------------------------------------------------------------------------------------------
% ---------------- Modification des titres de niveau 1,2 et 3 --------------------
% -------------------------------------------------------------------------------------------------
\titleformat
{\section} % command
%[display] % shape
{\Large} % format
{{\color{red}\uuline{\color{black}}}} % label
{0ex} % sep
{\sectionred} % before-code
[] % after-code

\titleformat
{\subsection}%[display] % shape
{\Large} % format
{{\color{ForestGreen}\uuline{\color{black}}}} % label
{0ex} % sep
{\sectiongreen} % before-code
[] % after-code

\titleformat
{\subsubsection} % command
%[display] % shape
{\large} % format
{{\color{NavyBlue}\uuline{\color{black}}}} % label
{0ex} % sep
{\sectionblue} % before-code
[] % after-code

% -------------------------------------------------------------------------------------------------
% ---------------- Début du corps du document ------------------------------------
% -------------------------------------------------------------------------------------------------

On se place dans un référentiel supposé galiléen, où le principe d'inertie est vérifié, avec un repère d'espace fixe et un repère de temps newtonien. Les objets sont supposés ponctuels.

A est le point de début du saut et B le point le plus haut du saut. On cherche une altitude, en l'occurence $Z_B$.

Comme on néglige les forces de frotement : $\Delta E_M = 0$.

Or, $\Delta E_M = {\color{orange}\Delta E_C} + {\color{ForestGreen}\Delta E_{pp}}$.

Donc : ${\color{orange}\Delta E_C} + {\color{ForestGreen}\Delta E_{pp}} = 0$.

Or, ${\color{ForestGreen}\Delta E_{pp}} = {\color{green}E_{pp, B} - E_{pp, A}}$ et ${\color{orange}\Delta E_C} = {\color{red}E_{C, B} - E_{C, A}}$.

Or, ${\color{green}E_{pp} = mg \times Z}$ et ${\color{red}E_{C} = \frac{1}{2}mv^2}$.

On obtient l'équation suivante : $({\color{red}\frac{1}{2}mv_B^2 - \frac{1}{2}mv_A^2}) + ({\color{green}mg \times Z_B - mg \times Z_A}) = 0$.

Or, $\frac{1}{2}mv_B^2 = 0$ car au point B, la vitesse et nulle. De plus, $mg \times Z_A = 0$, car l'altitude de référence, que l'on suppose nulle se trouve au point A.

On obtient : ${\color{red}-\frac{1}{2}mv_A^2} + {\color{green}mg \times Z_B} = 0$.

On fait passer $-\frac{1}{2}mv_A^2$ de l'autre côté de l'équation : ${\color{red}m}g \times Z_B = \frac{1}{2}{\color{red}m}v_A^2$.

On remarque les deux expressions de l'équation ont une grandeur en commun, $m$. On peut donc simplifier en l'enlevant pour obtenir : $g \times Z_B = \frac{1}{2}v_A^2$.

Il ne reste plus qu'à isoler la grandeur recherchée, ici $Z_B$ : $Z_B = \frac{1}{2}v_A^2 \times \frac{1}{g} = \frac{1}{2g}v_A^2 = \frac{v_A^2}{2g} $

APP.N : $Z_B = \frac{v_A^2}{2g} = \frac{(30 \:km/h)^2}{2 \times 9.81}$.

Mais notre valeur de vitesse est en km/h. Pour passer de km/h en m/s, on divise simplement par 3.6 ! (Pour passer de m/s en km/h, on multiplie donc par 3.6). On obtient donc :

$ Z_B = \frac{(\frac{30}{3.6})^2}{2 \times 9.81} = 3.5$\: m
\end{document}
