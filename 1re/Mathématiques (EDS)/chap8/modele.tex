\documentclass[a4paper,twoside,10pt,french,twocolumn]{scrartcl}
\usepackage{amsthm}
\usepackage{amsmath}
\usepackage[T1]{fontenc}
\usepackage{graphicx}
\usepackage{titlesec}
\usepackage{ulem}
\usepackage[dvipsnames]{xcolor}
\usepackage{color}
\usepackage{amssymb}
\usepackage{caption}
\usepackage[mathscr]{euscript}
\usepackage{babel} 
\usepackage{helvet} 
\usepackage[utf8]{inputenc}
\usepackage[T1]{fontenc}
\usepackage{geometry}
\usepackage{pgfplots}
\pgfplotsset{compat=1.15}
\usepackage{mathrsfs}
\usetikzlibrary{arrows}
\title{Les Suites}
\date{}
 \renewcommand*\familydefault{\sfdefault} 
\geometry{a4paper}
% -------------------------------------------------------------------------------------------------
% ----------- Création des commandes de couleur pour les titres ----------
% -------------------------------------------------------------------------------------------------
\newcommand{\sectionred}[1]{{\color{red}{\uuline{\color{black}#1}}}}
\newcommand{\sectiongreen}[1]{{\color{ForestGreen}{\uuline{\color{black}#1}}}}
\newcommand{\sectionblue}[1]{{\color{NavyBlue}{\uuline{\color{black}#1}}}}
\begin{document}
\maketitle
% -------------------------------------------------------------------------------------------------
% ---------------- Modification des titres de niveau 1,2 et 3 --------------------
% -------------------------------------------------------------------------------------------------
\titleformat
{\section} % command
%[display] % shape
{\Large} % format
{{\color{red}\uuline{\color{black}\thesection -}}} % label
{0ex} % sep
{\sectionred} % before-code
[] % after-code

\titleformat
{\subsection}%[display] % shape
{\Large} % format
{{\color{ForestGreen}\uuline{\color{black}\thesubsection -}}} % label
{0ex} % sep
{\sectiongreen} % before-code
[] % after-code

\titleformat
{\subsubsection} % command
%[display] % shape
{\large} % format
{{\color{NavyBlue}\uuline{\color{black}\thesubsubsection -}}} % label
{0ex} % sep
{\sectionblue} % before-code
[] % after-code

% -------------------------------------------------------------------------------------------------
% ---------------- Début du corps du document ------------------------------------
% -------------------------------------------------------------------------------------------------
\section{G\'en\'eralit\'es}
\subsection{G\'en\'eration}
\subsubsection{Formule\:Explicite}
Exemple : $U_n = 8n +2$
\subsubsection{Formule\:par\:R\'ecurrence}
Exemple : $U_{n+1} = U_n +2$
\subsection{Repr\'esentation}
On met n en abscice et $U_n$ en ordonn\'ee. On ne relie pas les points car U n'est d\'efinie que sur les nombres r\'eels.
\definecolor{ududff}{rgb}{0.30196078431372547,0.30196078431372547,1}
\definecolor{xdxdff}{rgb}{0.49019607843137253,0.49019607843137253,1}
\begin{tikzpicture}[line cap=round,line join=round,>=triangle 45,x=1cm,y=1cm]
\begin{axis}[
x=1cm,y=1cm,
axis lines=middle,
ymajorgrids=true,
xmajorgrids=true,
xmin=-2.5,
xmax=3.25,
ymin=-1.2400000000000002,
ymax=6.373333333333335,
xtick={-3,-2,...,9},
ytick={-1,0,...,6},]
\clip(-2.5,-1.24) rectangle (3.25,6.373333333333335);
\draw [->,line width=2pt] (0,0) -- (1,0);
\draw [->,line width=2pt] (0,0) -- (0,1);
\begin{scriptsize}
\draw[color=black] (0.5533333333333319,0.22) node {$i$};
\draw[color=black] (-0.006666666666668181,0.66) node {$j$};
\draw [fill=xdxdff] (0,0) circle (2.5pt);
\draw[color=xdxdff] (0.12666666666666518,0.22) node {$O$};
\draw [fill=ududff] (0,2) circle (2.5pt);
\draw [fill=ududff] (1,3) circle (2.5pt);
\draw [fill=ududff] (2,4) circle (2.5pt);
\draw [fill=ududff] (3,5) circle (2.5pt);
\end{scriptsize}
\end{axis}
\end{tikzpicture}
\section{Suites\:arithm\'etiques}
\subsection{D\'efinition}
Une suite est ``arithm\'etique'' si l'on ajoute toujours le m\^eme nombre (not\'e $r$) \`a un terme pour obtenir le suivant.
\subsection{Formule\:par\:r\'ecurrence}
$U_{n+1} = U_n +r$
\subsection{Formule\:explicite}
$U_n = U_p + (n-p) \times r$

Si $U_p = U_0$ : $U_n = U_0 + nr$.
\section{Suites\:g\'eom\'etriques}
\subsection{D\'efinition}
Une suite est ``g\'eom\'etrique'' si l'on multiplie toujours par le m\^eme nombre (not\'e $q$) \` un terme pour obtenir le suivant.
\subsection{Formule\:par\:r\'ecurrence}
$U_{n+1} = U_n \times q$
\subsection{Formule\:explicite}
$U_n = U_p \times q^{n-p}$

Si $U_p = U_0$ : $U_n = U_0 \times q^{n}$.
\section{Sommes}
\subsection{n\:premiers\:nombres\:entiers}
$S = 1 + 2 + 3 + \dots + n$

$S = \frac{n(n+1)}{2}$
\subsection{n\:premiers\:termes\:d'une\:suite\:arithm\'etique}
\subsubsection{de\:premier\:terme\:$U_0$}
$ S = U_0(n+1) + r\frac{n(n+1)}{2}$
\subsubsection{de\:premier\:terme\:$U_x$}
$ S = U_x(n+1-x) + r\frac{(n-x)(n+1-x)}{2}$
\subsection{n\:premi\`eres\:puissances\:d'un\:nombre\:entiers}
$S = 1 +q+q^2+\dots+q^n= \frac{1-q^{n+1}}{1-q}$
\subsection{n\:premiers\:termes\:d'une\:suite\:g\'eom\'etrique}
\subsubsection{de\:premier\:terme\:$U_0$}
$S = U_0 + U_1 + U_2 + \dots + U_n= U_0\frac{1-q^{n+1}}{1-q}$
\end{document}
