\documentclass[a4paper,twoside,10pt,french,twocolumn]{scrartcl}
\usepackage[T1]{fontenc}
\usepackage{titlesec}
\usepackage{ulem}
\usepackage[dvipsnames]{xcolor}
\usepackage{color}
\usepackage{babel} 
\usepackage{helvet} 
\usepackage[utf8]{inputenc}
\usepackage[T1]{fontenc}
\usepackage{geometry}
\title{Application de la dérivation et Variable aléatoire}
\date{}
 \renewcommand*\familydefault{\sfdefault} 
\geometry{a4paper}
% -------------------------------------------------------------------------------------------------
% ----------- Création des commandes de couleur pour les titres ----------
% -------------------------------------------------------------------------------------------------
\newcommand{\sectionred}[1]{{\color{red}{\uuline{\color{black}#1}}}}
\newcommand{\sectiongreen}[1]{{\color{ForestGreen}{\uuline{\color{black}#1}}}}
\newcommand{\sectionblue}[1]{{\color{NavyBlue}{\uuline{\color{black}#1}}}}
\begin{document}
% -------------------------------------------------------------------------------------------------
% ---------------- Modification des titres de niveau 1,2 et 3 --------------------
% -------------------------------------------------------------------------------------------------
\titleformat
{\section} % command
%[display] % shape
{\Large} % format
{{\color{red}\uuline{\color{black}\thesection -}}} % label
{0ex} % sep
{\sectionred} % before-code
[] % after-code

\titleformat
{\subsection}%[display] % shape
{\Large} % format
{{\color{ForestGreen}\uuline{\color{black}\thesubsection -}}} % label
{0ex} % sep
{\sectiongreen} % before-code
[] % after-code

\titleformat
{\subsubsection} % command
%[display] % shape
{\large} % format
{{\color{NavyBlue}\uuline{\color{black}\thesubsubsection -}}} % label
{0ex} % sep
{\sectionblue} % before-code
[] % after-code

% -------------------------------------------------------------------------------------------------
% ---------------- Début du corps du document ------------------------------------
% -------------------------------------------------------------------------------------------------
\section{Application\:de\:la\:dérivation}
\subsection{A\:la\:variation}
\begin{enumerate}
 \item Si $f'(x) < 0$ pour $x \in I$ alors f est décroissante sur $I$
 \item Si $f'(x) > 0$ pour $x \in I$ alors f est croissante sur $I$
  \item Si $f'(x) = 0$ pour $x \in I$ alors f est constante sur $I$
\end{enumerate}
Les récirproques sont vraies.
\subsection{Aux\:extremums}
\begin{enumerate}
 \item Si $f(x)$ est un extremum local de $f$, alors $f'(x) = 0$. La réciproque est fausse, pour $f(x)=x^3$ par exemple
 \item $f(x)$ est un extremum local de $f$ si $f'(x) = 0$ en changeant de signe (+;0;- ou -;0;+)
\end{enumerate}
\subsection{Aux\:fonctions\:polynômes}
\begin{enumerate}
 \item Si $a>0$, f est décroissante sur $]-\infty; \frac{-b}{2a}[$ et croissante sur $]\frac{-b}{2a};+\infty[$. $f(\frac{-b}{2a})$ est alors un minimum.
 \item Si $a<0$, f est croissante sur $]-\infty; \frac{-b}{2a}[$ et décroissante sur $]\frac{-b}{2a};+\infty[$. $f(\frac{-b}{2a})$ est alors un maximum.
\end{enumerate}
\section{Variable\:aléatoire}
\subsection{Définition}
Avec $a$ un réel quelconque :
\begin{enumerate}
 \item $\{X=a\}$ est l'évènement $X$ prend la valeur de a.
 \item $P(X=a)$ la probabilité que $X$ prennent la valeur a.
\end{enumerate}
\subsection{Lois\:de\:probabilité}
\subsubsection{Exemple}
\begin{tabular}{|l|l|l|l|}\hline
$a$ & 1 & 2 & 3\\\hline
$P(X=a)$ & $\frac{1}{3}$ & $\frac{1}{2}$ & $\frac{1}{6}$\\\hline
\end{tabular}
\subsubsection{Remarque}
On remarque que la somme des probabilité est toujours égale à 1
\subsection{Paramètres}
$x_1;x_2...x_n$ sont des valeurs que peut prendre $a$ et $p_1;p_2...p_n$ leur probabilité respectives.
\subsubsection{Espérence\:(=\:moyenne)}
$E(X) = x_1 \times p_1 + x_2 \times p_2 +...+ x_n \times p_n$
\subsubsection{Variance}
$V(X) = p_1(x_1-E(X))^2+p_2(x_2-E(X))^2+...+p_n(x_n-E(X))^2$

$V(X) = (\sigma(X))^2$
\subsubsection{Écart-type}
Elle donne une estimation de la dispertion des valeurs de la variable autour de l'espérence.

$\sigma(X)=\sqrt{V(X)}$

\end{document}
