\documentclass[a4paper,twoside,10pt,french, twocolumn]{scrartcl}
\usepackage[T1]{fontenc}
\usepackage{titlesec}
\usepackage{ulem}
\usepackage[dvipsnames]{xcolor}
\usepackage{color}
\usepackage{babel} 
\usepackage{helvet} 
\usepackage[utf8]{inputenc}
\usepackage[T1]{fontenc}
\usepackage{geometry}
\usepackage{pst-all}
\usepackage{auto-pst-pdf} %Pour compiler les pstricks avec PDFLaTeX
\title{Chapitre 4 : Produits scalaires}
\date{}
 \renewcommand*\familydefault{\sfdefault} 
\geometry{a4paper}
% -------------------------------------------------------------------------------------------------
% ----------- Création des commandes de couleur pour les titres ----------
% -------------------------------------------------------------------------------------------------
\newcommand{\sectionred}[1]{{\color{red}{\uuline{\color{black}#1}}}}
\newcommand{\sectiongreen}[1]{{\color{ForestGreen}{\uuline{\color{black}#1}}}}
\newcommand{\sectionblue}[1]{{\color{NavyBlue}{\uuline{\color{black}#1}}}}
\begin{document}
\maketitle
% -------------------------------------------------------------------------------------------------
% ---------------- Modification des titres de niveau 1,2 et 3 --------------------
% -------------------------------------------------------------------------------------------------
\titleformat
{\section} % command
%[display] % shape
{\Large} % format
{{\color{red}\uuline{\color{black}\thesection -}}} % label
{0ex} % sep
{\sectionred} % before-code
[] % after-code

\titleformat
{\subsection}%[display] % shape
{\Large} % format
{{\color{ForestGreen}\uuline{\color{black}\thesubsection -}}} % label
{0ex} % sep
{\sectiongreen} % before-code
[] % after-code

\titleformat
{\subsubsection} % command
%[display] % shape
{\large} % format
{{\color{NavyBlue}\uuline{\color{black}\thesubsubsection -}}} % label
{0ex} % sep
{\sectionblue} % before-code
[] % after-code

% -------------------------------------------------------------------------------------------------
% ---------------- Début du corps du document ------------------------------------
% -------------------------------------------------------------------------------------------------

\section{Définition\:avec\:le\:cosinus}
$\overrightarrow{AB}.\overrightarrow{AC} = AB \times AC \times \cos (\widehat{BAC})$
\section{Avec\:le\:projeté\:orthogonal}
Pour calculer $\overrightarrow{AB}.\overrightarrow{AC}$, on projette $C$ sur $(AB)$. Ainsi, $\overrightarrow{AB}.\overrightarrow{AC} = \overrightarrow{AB}.\overrightarrow{AH}$. Les 2 vecteurs sont colinéaires, on peut calculer le produit scalaire avec le produit de leur norme.

\psset{unit=1.2cm,arrowsize=7pt}
\psset{unit=1cm,arrowsize=7pt}
\begin{pspicture}(0,0)(4,2.5)
  \psline{->}(0,0)(2,2)
  \psline{->}(0,0)(4,0)
  \psline[linestyle=dashed](2,0.)(2,2.5)
  \rput(-0.3,0){$A$}\rput(4.3,0){$B$}
  \rput(1.6,2){$C$}\rput(2,-0.3){$H$}
  \psline(1.8,0)(1.8,0.2)(2,0.2)
\end{pspicture}
\section{Dans\:un\:repère\:orthonormé}
Dans un repère orthonormé, $\overrightarrow{u}.\overrightarrow{v} = x \times u' + y \times y'$ avec $x$ et $x'$ les coordonnées x respectives des 2 vecteurs et $y$ et $y'$ leurs coordonnées y respectives.
\section{Propriétés}
\subsection{Géométriques}
\begin{itemize}
 \item $\overrightarrow{u}.\overrightarrow{v} = \overrightarrow{v}.\overrightarrow{u}$
 \item $\overrightarrow{u}.(\overrightarrow{v}+\overrightarrow{w}) = \overrightarrow{u}.\overrightarrow{v} + \overrightarrow{u}.\overrightarrow{w}$
 \item $\overrightarrow{u}.\overrightarrow{v} = 0 \Leftrightarrow \overrightarrow{u} \perp \overrightarrow{v}$
\end{itemize}
\subsection{Algébriques}
\begin{itemize}
 \item $(\overrightarrow{u}+\overrightarrow{v})^2 = \overrightarrow{u}^2 + 2\overrightarrow{v}.\overrightarrow{u} + \overrightarrow{v}^2$
 \item $(\overrightarrow{u}-\overrightarrow{v})^2 = \overrightarrow{u}^2 - 2\overrightarrow{v}.\overrightarrow{u} + \overrightarrow{v}^2$
 \item $\overrightarrow{u}^2-\overrightarrow{v}^2 = (\overrightarrow{u}+\overrightarrow{v}).(\overrightarrow{u}-\overrightarrow{v})$
\end{itemize}
\section{Avec\:les\:normes}
\begin{itemize}
 \item $\overrightarrow{u}.\overrightarrow{v} = \frac{1}{2} ( \Arrowvert \overrightarrow{u}+\overrightarrow{v}\Arrowvert^2 -\Arrowvert\overrightarrow{u}\Arrowvert^2 -\Arrowvert\overrightarrow{v}\Arrowvert^2)$
 \item $\overrightarrow{u}.\overrightarrow{v} = \frac{1}{2} ( \Arrowvert\overrightarrow{u}\Arrowvert^2+\Arrowvert\overrightarrow{v}\Arrowvert^2 -\Arrowvert\overrightarrow{u} -\overrightarrow{v}\Arrowvert^2)$
\end{itemize}
\section{Relation\:de\:Chasles}
\subsection{Propriété}
$\overrightarrow{AC}=\overrightarrow{AB}+\overrightarrow{BC}$
\subsection{Exemple}
$\overrightarrow{DE}.\overrightarrow{AC} = (\overrightarrow{DA}+\overrightarrow{AE}).(\overrightarrow{AB}+\overrightarrow{BC})$
\psset{unit=1cm,arrowsize=7pt}
\begin{pspicture}(0,3.5)(4,4)
  \pspolygon(0,0)(5,0)(5,3)(0,3)
  \psplot{-0.3}{3}{-6 5 div x mul 3 add}
  \psplot{-0.3}{5.3}{3 5 div x mul}
  %
  \rput(0,-0.2){$A$}
  \rput(5.2,-0.2){$B$}
  \rput(5.,3.2){$C$}
  \rput(0.1,3.2){$D$}
  \rput(2.4,-0.2){$E$}
  %
\end{pspicture}
\end{document}
