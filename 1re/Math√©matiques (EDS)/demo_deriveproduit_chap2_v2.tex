\documentclass[a4paper,twoside,10pt,twocolumn,french]{scrartcl}
\usepackage[T1]{fontenc}
\usepackage{titlesec}
\usepackage{ulem}
\usepackage[dvipsnames]{xcolor}
\usepackage{color}
\usepackage{babel} 
\usepackage{helvet} 
\usepackage[utf8]{inputenc}
\usepackage[T1]{fontenc}
\usepackage{geometry}
\usepackage[fleqn,tbtags]{mathtools}
\title{Démonstration de  $(uv)' = u'v+uv'$}
\date{}
 \renewcommand*\familydefault{\sfdefault} 
\geometry{a4paper}
% -------------------------------------------------------------------------------------------------
% ----------- Création des commandes de couleur pour les titres ----------
% -------------------------------------------------------------------------------------------------
\newcommand{\sectionred}[1]{{\color{red}{\uuline{\color{black}#1}}}}
\newcommand{\sectiongreen}[1]{{\color{ForestGreen}{\uuline{\color{black}#1}}}}
\newcommand{\sectionblue}[1]{{\color{NavyBlue}{\uuline{\color{black}#1}}}}
\begin{document}
\maketitle
% -------------------------------------------------------------------------------------------------
% ---------------- Modification des titres de niveau 1,2 et 3 --------------------
% -------------------------------------------------------------------------------------------------
\titleformat
{\section} % command
%[display] % shape
{\Large} % format
{{\color{red}\uuline{\color{black}\thesection -}}} % label
{0ex} % sep
{\sectionred} % before-code
[] % after-code

\titleformat
{\subsection}%[display] % shape
{\Large} % format
{{\color{ForestGreen}\uuline{\color{black}\thesubsection -}}} % label
{0ex} % sep
{\sectiongreen} % before-code
[] % after-code

\titleformat
{\subsubsection} % command
%[display] % shape
{\large} % format
{{\color{NavyBlue}\uuline{\color{black}\thesubsubsection -}}} % label
{0ex} % sep
{\sectionblue} % before-code
[] % after-code

% -------------------------------------------------------------------------------------------------
% ---------------- Début du corps du document ------------------------------------
% -------------------------------------------------------------------------------------------------

\section{Remarques\:préliminaires}
$u$ et $v$ sont dérivable sur $I$ et $ a \in I$
\\
\section{Taux\:de\:variation}
On pose d'abord le taux de variation :   \begin{align} T_a(h) & = \frac{f(a+h)-f(a)}{h} \\
& = \frac{u(a+h)\times v(a+h) - u(a)\times v(a)}{h} \end{align} 
\\
\section{Ajout\:et\:soustraction}
Pour la suite de la démonstration, il faut ajouter et soustraire en même temps $u(a+h) \times v(a) $. On a donc : $$T_a(h) = \frac{\colorbox{Yellow}{$u(a+h$)}\times v(a+h) {\color{red}- {\colorbox{Yellow}{$u(a+h$)} \times v(a)}} {\color{ForestGreen}+ u(a+h) \times \colorbox{SpringGreen}{$v(a)$} }- u(a)\times \colorbox{SpringGreen}{$v(a)$}}{h} $$
\section{Factorisation}
On peut désormais procéder à la factorisation comme ceci : $$T_a(h) = \frac{\colorbox{Yellow}{$u(a+h$)}\times (v(a+h){\color{red}-v(a)})+\colorbox{SpringGreen}{$v(a)$} \times ({\color{ForestGreen}u(a+h)}-u(a))}{h} $$
\section{Séparation\:de\:l'expression\:en\:2} 
On sépare l'expression : $$T_a(h) = \frac{u(a+h)\times (v(a+h){\color{red}-v(a)})}{h}+ \frac{v(a) \times ({\color{ForestGreen}u(a+h)}-u(a))}{h} $$
\section{Simplification}
On simplifie en enlevant la division par h pour les facteurs $u(a+h)$ et $v(a)$. On a donc : $$T_a(h) = {\color{OrangeRed}u(a+h)}\times {\color{Orange}\frac{(v(a+h)-v(a))}{h}}+ {\color{Orchid}v(a)} \times {\color{NavyBlue}\frac{(u(a+h)-u(a))}{h}} $$
\section{On\:fait\:tendre\:h\:vers\:0}
On fait tendre h vers 0 pour trouver $T_a(h)$. La limite quand $h \rightarrow 0$ est donc : $$T_a(h) = {\color{OrangeRed}u(a)} \times {\color{Orange}v'(a)} + {\color{Orchid}v(a)} \times {\color{NavyBlue}u'(a)}$$
\section{Conclusion}
%Donc la limite quand $h \rightarrow 0$ est : $$T_a(h) = u(a) \times v'(a) + v(a) \times u'(a)$$
Donc $f$ est dérivable en $a$ et $f'(a) = u'(a)v(a) + u(a)v'(a)$ Ceci est valable pour tout $a$ de $I$, alors $f$ est dérivable en sur $I$ et $f'=u'v+uv'$
\end{document}